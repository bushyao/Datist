%% Generated by Sphinx.
\def\sphinxdocclass{report}
\documentclass[a4paper,10pt,english]{sphinxmanual}
\ifdefined\pdfpxdimen
   \let\sphinxpxdimen\pdfpxdimen\else\newdimen\sphinxpxdimen
\fi \sphinxpxdimen=.75bp\relax


\catcode`^^^^00a0\active\protected\def^^^^00a0{\leavevmode\nobreak\ }
\usepackage{cmap}
\usepackage{fontspec}
\usepackage{amsmath,amssymb,amstext}
\usepackage{polyglossia}
\setmainlanguage{english}

\usepackage[Sonny]{fncychap}
\usepackage[dontkeepoldnames]{sphinx}

\usepackage{geometry}

% Include hyperref last.
\usepackage{hyperref}
% Fix anchor placement for figures with captions.
\usepackage{hypcap}% it must be loaded after hyperref.
% Set up styles of URL: it should be placed after hyperref.
\urlstyle{same}
\addto\captionsenglish{\renewcommand{\contentsname}{Contents:}}

\addto\captionsenglish{\renewcommand{\figurename}{图}}
\addto\captionsenglish{\renewcommand{\tablename}{表}}
\addto\captionsenglish{\renewcommand{\literalblockname}{列表}}

\addto\captionsenglish{\renewcommand{\literalblockcontinuedname}{continued from previous page}}
\addto\captionsenglish{\renewcommand{\literalblockcontinuesname}{continues on next page}}

\def\pageautorefname{page}

\setcounter{tocdepth}{0}


\documentclass[UTF8]{ctexart} 
\usepackage{CJKutf8}


\title{地震相关专业名词与术语规范}
\date{2018 年 01 月 06 日}
\release{1.0}
\author{i}
\newcommand{\sphinxlogo}{\vbox{}}
\renewcommand{\releasename}{发布}
\makeindex

\begin{document}

\maketitle
\sphinxtableofcontents
\phantomsection\label{\detokenize{index::doc}}



\chapter{一、原理篇}
\label{\detokenize{index:id1}}\label{\detokenize{index:id2}}

\section{1.什么是地震?}
\label{\detokenize{index:id3}}
答:广义地说,地震是地球表层的震动;根据震动性质不同可分为三类: \sphinxcode{天然地震} 指自然界发生的地震现象; \sphinxcode{人工地震}  由爆破、核试验等人为因素引起的地面震动;\sphinxcode{脉动} 由于大气活动、海浪冲击等原因引起的地球表层的经常性微动。狭义而言,人们平时所说的地震是指能够形成灾害的天然地震。


\section{2.天然地震有几种类型?}
\label{\detokenize{index:id4}}
答:天然地震按成因不同主要有三种类型: \sphinxcode{构造地震} 由地下深处岩层错动、破裂所造成的地震。这类地震发生的次数最多,约占全球地震数的90%以上,破坏力也最大。 \sphinxcode{火山地震} 由于火山作用,如岩浆活动、气体爆炸等引起的地震。它的影响范围一般较小,发生得也较少,约占全球地震数的7%。 \sphinxcode{陷落地震} 由于地层陷落引起的地震。例如,当地下岩洞或矿山采空区支撑不住顶部的压力时,就会塌陷引起地震。这类地震更少,大约不到全球地震数的3%,引起的破坏也较小。


\section{3.构造地震是怎样发生的?}
\label{\detokenize{index:id5}}
答:通常,我们所说的地震是指 \sphinxcode{构造地震} 。它是怎样发生的呢?这就要从地球的内部构造说起。地球是一个平均半径约为6370千米的多层球体,最外层的地壳相当薄,平均厚度约为33千米,它与地幔(厚约2900千米)的最上层共同形成了厚约100千米的岩石圈。在构造力的作用下,当岩石圈某处岩层发生突然破裂、错动时,便把长期积累起来的能量在瞬间急剧释放出来,巨大的能量以地震波的形式由该处向四面八方传播出去,直到地球表面,引起地表的震动,便造成地震。


\section{4.什么是断层,它与地震有关吗?}
\label{\detokenize{index:id6}}
答:断层是地下岩层沿一个破裂面或破裂带两侧发生相对位错的现象。地震往往是由断层活动引起的,是断层活动的一种表现,所以地震与断层的关系十分密切。断层一般在中上地壳最为明显,有的直接出露地表,有的则隐伏在地下;它们的规模也各不相同。岩石发生相对位移的破裂面称为断层面;根据断层面两盘运动方式的不同,大致可分为 \sphinxcode{正断层} (上盘相对下滑)、\sphinxcode{逆断层} (上盘相对上冲)、\sphinxcode{平移断层} (两盘沿断层走向相对水平错动)三种类型。与地震发生关系最为密切的是在现代构造环境下曾有活动的那些断层,即活断层。


\section{5.全球每年发生多少地震?}
\label{\detokenize{index:id7}}
答:地球上每年约发生 \sphinxcode{500多万次}  地震,也就是说,每天要发生上万次地震。不过,它们之中绝大多数太小或离我们太远,人们感觉不到。真正能对人类造成严重危害的地震,全世界每年大约有一二十次;能造成唐山、汶川这样特别严重灾害的地震,每年大约有一两次。
人们感觉不到的地震,须用地震仪才能记录下来;不同类型的地震仪能记录不同强度、不同远近的地震。目前世界上运转着数以千计的各种地震仪器,日夜监测着地震的动向。


\section{6.什么叫地震波,它有哪些类型?}
\label{\detokenize{index:id8}}
答:地震发生时,地下岩层断裂错位释放出巨大的能量,激发出一种向四周传播的弹性波,这就是地震波。地震波主要分为体波和面波。体波可以在三维空间中向任何方向传播,又可分为纵波和横波。
\begin{quote}

\sphinxcode{纵波} 振动方向与波的传播方向一致的波,传播速度较快,到达地面时人感觉颠动,物体上下跳动。
\sphinxcode{横波} 振动方向与波的传播方向垂直,传播速度比纵波慢,到达地面时人感觉摇晃,物体会来回摆动。
\sphinxcode{面波} 当体波到达岩层界面或地表时,会产生沿界面或地表传播的幅度很大的波,称为面波。面波传播速度小于横波,所以跟在横波的后面。
\end{quote}


\section{7.什么叫震源? 什么叫震中,它是怎样确定的?}
\label{\detokenize{index:id9}}
答:地球内部直接产生破裂的地方称为 \sphinxcode{震源} ,它是一个区域,但研究地震时常把它看成一个点。地面上正对着震源的那一点称为 \sphinxcode{震中} ,它实际上也是一个区域。
根据地震仪记录测定的震中称为 \sphinxcode{微观震中} ,用经纬度表示;根据地震宏观调查所确定的震中称为宏观震中,它是极震区(震中附近破坏最严重的地区)的几何中心,也用经纬度表示。由于方法不同,宏观震中与微观震中往往并不重合。1900年以前没有仪器记录时,地震的震中位置都是按破坏范围而确定的 \sphinxcode{宏观震中} 。


\section{8.什么叫震中距,如何划分地震的远近?}
\label{\detokenize{index:id10}}
答:从震中到地面上任何一点的距离叫做震中距。同一个地震在不同的距离上观察,远近不同,叫法也不一样。对于观察点而言,震中距大于1000千米的地震称为远震,震中距在100~1000千米的称为近震,震中距在100千米以内的称为地方震。例如,汶川地震对于300多千米处的重庆而言为近震;而对千里之外的北京而言,则为远震。


\section{9.什么叫震源深度?}
\label{\detokenize{index:id11}}
答:从震源到地面的距离叫做震源深度。震源深度在60千米以内的地震为浅源地震,震源深度超过300千米的地震为深源地震,震源深度为60~300千米的地震为中源地震。同样强度的地震,震源越浅,所造成的影响或破坏越重。我国绝大多数地震为浅源地震。


\section{10.什么是震级,它是怎样测定的?}
\label{\detokenize{index:id12}}
答:震级是衡量地震本身大小的一把“尺子”,它与震源释放出来的弹性波能量有关。震级越高,表明震源释放的能量越大;震级相差一级,能量相差30多倍。
震级通常是通过地震仪记录到的地面运动的振动幅度来测定的,由于地震波传播路径、地震台台址条件等的差异,不同台站所测定的震级不尽相同,所以常常取各台的平均值作为一次地震的震级。
地震发生时,距震中较近的台站常会因为仪器记录振幅“出格”而难以确定震级,此时就必须借助更远的台站来测定。所以,地震过后一段时间对震级进行修订是常有的事。


\section{11.地震按震级大小可分为几类?}
\label{\detokenize{index:id13}}\begin{description}
\item[{答:地震按震级大小的划分大致如下:}] \leavevmode
\sphinxcode{弱震} 震级小于3级。如果震源不是很浅,这种地震人们一般不易觉察。
\sphinxcode{有感地震} 震级大于或等于3级、小于或等于4.5级。这种地震人们能够感觉到,但一般不会造成破坏。
\sphinxcode{中强震} 震级大于4.5级、小于6级,属于可造成损坏或破坏的地震,但破坏轻重还与震源深度、震中距等多种因素有关。
\sphinxcode{强震} 震级大于或等于6级,是能造成严重破坏的地震。其中震级大于或等于8级的又称为巨大地震。

\end{description}


\section{12.什么是地震烈度,它与震级有什么不同?}
\label{\detokenize{index:id14}}
答:地震烈度是衡量地震影响和破坏程度的一把“尺子” ,简称烈度。烈度与震级不同。震级反映地震本身的大小,只与地震释放的能量多少有关;而烈度则反映的是地震的后果,一次地震后不同地点烈度不同。打个比方,震级好比一盏灯泡的瓦数,烈度好比某一点受光亮照射的程度,它不仅与灯泡的功率有关,而且与距离的远近有关。因此,一次地震只有一个震级,而烈度则各地不同。
一般而言,震中地区烈度最高,随着震中距加大,烈度逐渐减小。例如,1976年唐山地震,震级为7.8级,震中烈度为Ⅺ度;受唐山地震影响,天津市区烈度为Ⅷ度,北京市多数地区烈度为Ⅵ度,再远到石家庄、太原等地烈度就更低了。


\section{13.地震烈度是怎样评定的?}
\label{\detokenize{index:id15}}
答:地震烈度是以人的感觉、器物反应、房屋等结构和地表破坏程度等进行综合评定的,反映的是一定地域范围内(如自然村或城镇部分区域)地震破坏程度的平均水平,须由科技人员通过现场调查予以评定。
一次地震后,一个地区的地震烈度会受到震级、震中距、震源深度、地质构造、场地条件等多种因素的影响。
用于说明地震烈度的等级划分、评定方法与评定标志的技术标准是地震烈度表,各国所采用的烈度表不尽相同。


\section{14.我国评定地震烈度的技术标准是什么?}
\label{\detokenize{index:id16}}
答:我国评定地震烈度的技术标准是《中国地震烈度表(1980)》,它将烈度划分为12度,其评定依据之一是:Ⅰ~Ⅴ度以地面上人的感觉为主;Ⅵ~Ⅹ度以房屋震害为主,人的感觉仅供参考;Ⅺ、Ⅻ度以房屋破坏和地表破坏现象为主。
按这个烈度表的评定标准,一般而言,烈度为Ⅲ~Ⅴ度时人们有感,Ⅵ度以上有破坏,Ⅸ~Ⅹ度破坏严重,Ⅺ度以上为毁灭性破坏。


\section{15.什么是烈度分布图?什么是烈度异常区?}
\label{\detokenize{index:id17}}
答:烈度分布图又叫做等震线图。震后调查结束后,将各烈度评定点的结果标示在适当比例尺的地图上,然后由高到低把烈度相同点的外包线(即等震线)勾画出来,便构成地震烈度分布图。
震中区的烈度称为震中烈度,唐山、汶川地震的震中烈度都达到Ⅺ度。一般而言,震中地区烈度最高,随着震中距加大,烈度逐渐减小。但是也存在局部地区的烈度高于或低于周边烈度的现象,如果这种烈度异常点连片出现,则可划分出一个局部的烈度异常区。
造成烈度异常的原因往往是场地条件:软弱场地易加重震害,形成高烈度异常区;坚硬场地则可减小震害,形成低烈度异常区。这就是地震破坏程度并非随震中距的加大而一致减小的原因。


\section{16.震源深度对震中烈度有影响吗?}
\label{\detokenize{index:id18}}
答:震源深度对地震的破坏程度影响很大。同样大小的地震,震源越浅,造成的破坏越重。据统计,当震源深度从20千米减小到10千米,或从10千米减小到5千米时,震中烈度均可提高1度。这常常是有些地震震级并不太高,但破坏较严重的原因之一。


\section{17.什么是地震带,世界上有几个主要地震带?}
\label{\detokenize{index:id19}}
答:地震带是地震集中分布的地带,在地震带内地震密集,在地震带外,地震分布零散。世界上主要有三大地震带:
环太平洋地震带 分布在太平洋周围,包括南北美洲太平洋沿岸和从阿留申群岛、堪察加半岛、日本列岛南下至我国台湾省,再经菲律宾群岛转向东南,直到新西兰。这里是全球分布最广、地震最多的地震带,所释放的能量约占全球的四分之三。
\sphinxcode{欧亚地震带}   从地中海向东,一支经中亚至喜马拉雅山,然后向南经我国横断山脉,过缅甸,呈弧形转向东,至印度尼西亚。另一支从中亚向东北延伸,至堪察加,分布比较零散。
\sphinxcode{海岭地震带} 分布在太平洋、大西洋、印度洋中的海岭地区(海底山脉)。


\section{18.什么是板块构造,它与地震活动有关吗?}
\label{\detokenize{index:id20}}
答:地球最上层包括地壳在内的约100千米范围的岩石圈并不完整,像是打碎了仍然连在一起的鸡蛋壳,这些大小不等、拼接在一起的岩石层称为板块,它们各自在上地幔内的软流层上“漂浮”、运移,有的板块会俯冲到地幔内数百千米深的地方。
地球上最大的板块有六块,分别是太平洋板块、欧亚板块、美洲板块、非洲板块、印度洋板块和南极洲板块。另外还有一些较小的板块,如菲律宾板块等。
把世界地震分布与全球板块分布相比较,可以明显看出两者非常吻合。据统计,全球有85\%的地震发生在板块边界上,仅有15\%的地震与板块边界的关系不那么明显。这就说明,板块运动过程中的相互作用,是引起地震的重要原因。


\section{19.什么是板缘地震?什么是板内地震?}
\label{\detokenize{index:id21}}
答:发生在板块边界上的地震叫板缘地震,环太平洋地震带上绝大多数地震属于此类;发生在板块内部的地震叫板内地震,如欧亚大陆内部(包括我国)的地震多属此类。板内地震除与板块运动有关,还要受局部地质环境的影响,其发震的原因与规律比板缘地震更复杂。


\section{20.我国为什么是多地震的国家?}
\label{\detokenize{index:id22}}
答:我国地处欧亚大陆东南部,位于环太平洋地震带和欧亚地震带之间,有些地区本身就是这两个地震带的组成部分。受太平洋板块、印度洋板块和菲律宾板块的挤压作用,我国地质构造复杂,地震断裂带十分发育,地震活动的范围广、强度大、频率高。在全球大陆地区的大地震中,约有四分之一至三分之一发生在我国。自1900年至20世纪末,我国已发生4¾ 级以上地震3800余次;其中,6~6.9级地震460余次,7~7.9级地震99次, 8级以上地震9次。


\section{21.我国地震主要分布在哪些地方?}
\label{\detokenize{index:id23}}
答:我国的地震活动主要分布在5个地区的23条地震带上,这5个地区是:
①台湾省及其附近海域;
②西南地区,包括西藏、四川中西部和云南中西部;
③西部地区,主要在甘肃河西走廊、青海、宁夏以及新疆天山南北麓;
④华北地区,主要在太行山两侧、汾渭河谷、阴山—燕山一带、山东中部和渤海湾;
⑤东南沿海地区,广东、福建等地。


\section{22.什么是“南北地震带”?}
\label{\detokenize{index:id24}}
答:从我国的宁夏,经甘肃东部、四川中西部直至云南,有一条纵贯中国大陆、大致呈南北走向的地震密集带,历史上曾多次发生强烈地震,被称为中国南北地震带。2008年5月12日汶川8.0级地震就发生在该带中南段。该带向北可延伸至蒙古境内,向南可到缅甸。


\section{23.什么叫地震活动的周期性?}
\label{\detokenize{index:id25}}
答:通过对历史地震和现今地震大量资料的统计,发现地震活动在时间上的分布是不均匀的:一段时间发生地震较多,震级较大,称为地震活跃期;另一段时间发生地震较少,震级较小,称为地震活动平静期;表现出地震活动的周期性。每个活跃期均可能发生多次7级以上地震,甚至8级左右的巨大地震。地震活动周期可分为几百年的长周期和几十年的短周期;不同地震带活动周期也不尽相同。


\section{24.什么是地震序列?}
\label{\detokenize{index:id26}}
答:一次中强以上地震前后,在震源区及其附近,往往有一系列地震相继发生;这些成因上有联系的地震就构成了一个地震序列。
根据地震序列的能量分布、主震能量占全序列能量的比例、主震震级和最大余震的震级差等,可将地震序列划分为主震-余震型、震群型、孤立型三类;根据有无前震,又可把地震序列分为主震-余震型、前震-主震-余震型、震群型三类。
由于强震发生后,往往还会有较大余震,甚至更大地震发生,所以震后还须防备强余震的袭击。


\section{25.什么是主震-余震型地震?}
\label{\detokenize{index:id27}}
答:主震-余震型地震的特点是:主震非常突出,余震十分丰富;最大地震所释放的能量占全序列的90\%以上;主震震级和最大余震相差0.7~2.4级。
有时,主震发生前先有一些前震出现,这种主震-余震型地震也叫前震-主震-余震型地震。例如1975年2月4日辽宁海城7.3级地震前,自2月1日起即突然出现小震活动,且其频度和强度都不断升高,于2月4日上午出现两次有感地震;主震于当日18时36分发生。


\section{26.什么是震群型地震?}
\label{\detokenize{index:id28}}
答:有两个以上大小相近的主震,余震十分丰富;主要能量通过多次震级相近的地震释放,最大地震所释放的能量占全序列的90\%以下;主震震级和最大余震相差0.7级以下。如1966年河北邢台地震即属此类,在3月8日~22日的15天内,先后发生6级以上地震5次,震级分别为7.2,6.8,6.7,6.2,6.0级。


\section{27.什么是孤立型地震?}
\label{\detokenize{index:id29}}
答:有突出的主震,余震次数少、强度低;主震所释放的能量占全序列的99.9\%以上;主震震级和最大余震相差2.4级以上。例如,1983年11月7日山东菏泽5.9级地震即属于此类,它的最大余震只有3级左右。


\section{28.我国地震灾害为什么严重?}
\label{\detokenize{index:id30}}
答:地震作为一种自然现象本身并不是灾害,但当它达到一定强度,发生在有人类生存的空间,且人们对它没有足够的抵御能力时,便可造成灾害。地震越强,人口越密,抗御能力越低,灾害越重。
我国恰恰在以上三方面都十分不利。首先,我国地震频繁,强度大,而且绝大多数是发生在大陆地区的浅源地震,震源深度大多只有十几至几十千米。其次,我国许多人口稠密地区,如台湾、福建、四川、云南等,都处于地震的多发地区,约有一半城市处于地震多发区或强震波及区,地震造成的人员伤亡十分惨重。第三,我国经济不够发达,广大农村和相当一部分城镇,建筑物质量不高,抗震性能差,抗御地震的能力低。
所以,我国地震灾害十分严重。20世纪内,我国已有50多万人死于地震,约占同期全世界地震死亡人数的一半。


\section{29.什么是地震的直接灾害?}
\label{\detokenize{index:id31}}
答:地震直接灾害是指由地震的原生现象,如地震断层错动,大范围地面倾斜、升降和变形,以及地震波引起的地面震动等所造成的直接后果。包括:
——建筑物和构筑物的破坏或倒塌;
——地面破坏,如地裂缝、地基沉陷、喷水冒砂等;
——山体等自然物的破坏,如山崩、滑坡、泥石流等;
——水体的振荡,如海啸、湖震等;
——其他如地光烧伤人畜等。
以上破坏是造成震后人员伤亡、生命线工程毁坏、社会经济受损等灾害后果最直接、最重要的原因。


\section{30.什么是地震的次生灾害?}
\label{\detokenize{index:id32}}
答:地震灾害打破了自然界原有的平衡状态或社会正常秩序从而导致的灾害,称为地震次生灾害。如地震引起的火灾、水灾,有毒容器破坏后毒气、毒液或放射性物质等泄漏造成的灾害等。
地震后还会引发种种社会性灾害,如瘟疫与饥荒。社会经济技术的发展还带来新的继发性灾害,如通信事故、计算机事故等。这些灾害是否发生或灾害大小,往往与社会条件有着更为密切的关系。


\section{31.地震火灾是怎样引起的?}
\label{\detokenize{index:id33}}
答:地震火灾多是因房屋倒塌后火源失控引起的。由于震后消防系统受损,社会秩序混乱,火势不易得到有效控制,因而往往酿成大灾。例如,1923年9月1日的日本关东地震发生在中午人们做饭之时,加之城内民居多为木质构造,震后立即引燃大火;而震裂的煤气管道和油库开裂溢出大量燃油,更助长了火势蔓延;由于消防设施瘫痪,大火竟燃烧了数天之久,烧毁房屋44万多座,造成10多万人死于地震火灾。


\section{32.地震水灾是怎样造成的?}
\label{\detokenize{index:id34}}
答:地震引起水库、江湖决堤,或是由于山体崩塌堵塞河道造成水体溢出等,都可能造成地震水灾。例如,1786年6月1日,我国四川省康定南发生7½级地震,大渡河沿岸出现大规模山崩,引起河流壅塞,形成堰塞湖;断流10日后,河道溃决,高数十丈的洪水汹涌而下,造成严重水患。


\section{33.地震海啸是怎样形成的,它对我国有危害吗?}
\label{\detokenize{index:id35}}
答:海啸是一种具有强大破坏力的海浪,除了地震以外,海底火山爆发或海底塌陷、滑坡等也能引起海啸。
由深海地震引起的海啸称为地震海啸。地震时海底地层发生断裂,部分地层出现猛烈上升或下沉,造成从海底到海面的整个水层发生剧烈“抖动”,这就是地震海啸。海啸形成后,大约以每小时数百千米的速度向四周海域传播,一旦进入大陆架,由于海水深度急剧变浅,使波浪高度骤然增加,有时可达二三十米,从而会对沿海地区造成严重灾难。
从历史记录和科学分析来看,远洋海啸对我国大陆沿海影响较小。但我国台湾沿海,尤其是台湾东部沿海,地震海啸的威胁不容忽视,尤其是由近海地震引起的局部海啸,应给予高度关注。


\section{34.什么是地震预报?}
\label{\detokenize{index:id36}}
答:地震预报是针对破坏性地震而言的,是在破坏性地震发生前作出预报,使人们可以防备。
地震预报三要素 地震预报要指出地震发生的时间、地点、震级,这就是地震预报的三要素。完整的地震预报这三个要素缺一不可。
地震预报按时间尺度可作如下划分:


\subsection{(1)长期预报}
\label{\detokenize{index:id37}}
是指对未来10年内可能发生破坏性地震的地域的预报。


\subsection{(2)中期预报}
\label{\detokenize{index:id38}}
是指对未来一二年内可能发生破坏性地震的地域和强度的预报。


\subsection{(3)短期预报}
\label{\detokenize{index:id39}}
是指对3个月内将要发生地震的时间、地点、震级的预报。


\subsection{(4)临震预报}
\label{\detokenize{index:id40}}
是指对10日内将要发生地震的时间、地点、震级的预报。


\section{35.地震能预报吗?}
\label{\detokenize{index:id41}}
答:地震预报是世界公认的科学难题,在国内外都处于探索阶段,大约从20世纪五六十年代才开始进行研究。我国地震预报的全面研究起步于1966年河北邢台地震,经过40多年的努力,取得了一定进展,曾经不同程度地预报过一些破坏性地震。
但是实践表明,目前所观测到的各种可能与地震有关的现象,都呈现出极大的不确定性;所作出的预报,特别是短临预报,主要是经验性的。
当前我国地震预报的水平和现状是:
——对地震前兆现象有所了解,但远远没有达到规律性的认识;
——在一定条件下能够对某些类型的地震,作出一定程度的预报;
——对中长期预报有一定的认识,但短临预报成功率还很低。


\section{36.什么是地震前兆?}
\label{\detokenize{index:id42}}
答:地震前自然界出现的可能与地震孕育、发生有关的各种征兆称作地震前兆。大体有两类:
微观前兆 人的感官不易觉察,须用仪器才能测量到的震前变化。例如,地面的变形,地球的磁场、重力场的变化,地下水化学成分的变化,小地震的活动等。
宏观前兆 人的感官能觉察到的地震前兆。它们大多在临近地震发生时出现。如井水的升降、变浑,动物行为反常,地声、地光等。
观测微观前兆是科学家的工作;而发现临近地震前的宏观前兆,则既要靠科学家,也要靠广大群众。由于宏观前兆往往在临近地震发生时出现,因此,了解它的特点,学会识别它们,对防震减灾有重要作用。


\section{37.地震微观前兆是怎样观测的?}
\label{\detokenize{index:id43}}
答:观测小地震的活动要使用地震仪;观测其他地震微观前兆则须使用前兆观测仪器,其种类很多。如观测和记录地壳形变的仪器有倾斜仪、自记水管仪、伸缩仪、水准仪、激光测距仪等;观测和记录地磁场变化的有磁变仪、核旋仪、地磁经纬仪等。观测地电、地应力、重力、水氡、水位、水质成分及其他微观前兆现象,也都有相应的仪器。


\section{38.震前地下水为什么会有异常变化?}
\label{\detokenize{index:id44}}
答:地震前地下岩层受力变形时,埋藏在含水岩层里的地下水的状况也会跟着改变。有时,含水层像饱含水的海绵一样,在受力时把水挤出来;有时,隔水层破裂,使原来分层流动的水掺和在一起;等等。这些变化都有可能通过井水、泉水等反映出来;这时,井或泉就成为人们观察地震前兆的“窗口”。


\section{39.震前地下水有哪些异常变化?}
\label{\detokenize{index:id45}}
答:①水位、水量的反常变化。如天旱时节井水水位上升,泉水水量增加;丰水季节水位反而下降或泉水断流。有时还出现井水自流、自喷等现象。
②水质的变化。如井水、泉水等变色、变味(如变苦、变甜)、变浑,有异味等。
③水温的变化。水温超过正常变化范围。
④其他。如翻花冒泡、喷气发响、井壁变形等。


\section{40.地下水异常一定与地震有关吗?}
\label{\detokenize{index:id46}}
答:不一定。由于地下水很容易受到环境的影响,所以它的异常变化并非一定与地震有关。影响地下水变化的因素有:气象因素,如干旱、降雨、气压变化等;地质因素,如非震的地质原因,改变了地下含水层的状态;人为因素,如用水量变化、地下工程活动、环境污染等。因此,发现异常后,要及时反映给地震部门去查明原因,做出判断。


\section{41.动物行为异常有哪些表现?}
\label{\detokenize{index:id47}}
答:多次震例表明,动物是观察地震前兆的“活仪器”,它们往往在震前出现各种反常行为,向人们预示灾难的临近。目前已发现有上百种动物震前有一定反常表现,其中异常反应比较普遍的有20多种,最常见的动物异常现象有:
\begin{quote}

\sphinxcode{惊恐反应 如大牲畜不进圈,狗狂吠,鸟或昆虫惊飞、非正常群迁等。}
\sphinxcode{抑制型异常 如行为变得迟缓,或发呆发痴,不知所措;或不肯进食等。}
\sphinxcode{生活习性变化 如冬眠的蛇出洞,老鼠白天活动不怕人,大批青蛙上岸活动等。}
\end{quote}


\section{42.动物行为异常一定与地震有关吗?}
\label{\detokenize{index:id48}}
答:不一定。因为引起动物反常现象的因素很多,例如天气变化、环境污染、饲养不当以及动物自身不适,如生病、怀孕等等。所以,动物有反常表现不一定就是地震前兆。另外,有时(特别是强震发生以后),人们情绪过分紧张,也可能在观察动物行为时出现错觉。因此,发现异常后不要惊慌,应及时反映给地震部门。


\section{43.什么是地声,它有什么特点?}
\label{\detokenize{index:id49}}
答:临近地震发生前,往往有声响自地下深处传来,这就是“地声”。地声一般出现在震前几分钟、几小时、几天或更早;以临震前几分钟出现得最多。
地声的声响与平日人们熟悉的声音不同且多种多样。如:“犹如列车从地下奔驰而来”“似采石放连珠炮般的声响”“类似于机器轰鸣声”“狂风呼啸声”“石头相互摩擦声”等等。但是,有时地声也不易与远处传来的风声、雷声、机器轰鸣声等相鉴别。


\section{44.地光有什么特点?}
\label{\detokenize{index:id50}}
答:地光也是临震前的一种宏观现象,我国已在多次地震前观测到,它们一般出现在临震前或震时,也有出现于震前数小时或更早的。
地光的颜色很多,有红、黄、蓝、白、紫等,有的也像电火光。它们的形状各异,有带状光、片形光、球状光、柱状光、火样光等。地光出现的时间一般很短,所以不易观测。鉴别地光也有一定难度,因为它的形状和颜色有时也与电焊光、闪电等有相似之处。


\section{45.你知道《地震监测设施和地震观测环境保护条例》吗?}
\label{\detokenize{index:id51}}
答:这个条例是在1994年1月10日由国务院颁布的,其目的是为了保证各类地震观测仪器正常工作,以取得可靠的数据,每个公民都应当自觉贯彻这个条例。
条例中明确规定的地震监测设施的保护范围是:
①地震台内的监测仪器设备、设施;
②地震台外的观测用山洞、仪器房、观测井(水点)、井房、观测线路、通信设施、供电设施、供水设施、专用填坝、专用道路、避雷装置及其附属设施;
③地震遥测台网接收中心的观测设备、中继站、遥测点用房等;地震专用测量标志、测量场地等。


\section{46.你知道地震预报应当由谁发布吗?}
\label{\detokenize{index:id52}}
答:面向社会发布地震预报是一件十分严肃的事情。
为了加强对地震预报的管理,规范发布地震预报的行为,1998年12月27日,中华人民共和国国务院颁发了《地震预报管理条例》,规定“国家对地震预报实行统一发布制度。”具体规定主要是:
全国性的地震长期预报和地震中期预报,由国务院发布。
省、自治区、直辖市行政区域内的地震长期预报、地震中期预报、地震短期预报和临震预报,由省、自治区、直辖市人民政府发布。
已经发布地震短期预报的地区,如果发现明显临震异常,在紧急情况下,当地市、县人民政府可以发布48小时之内的临震预报,并同时向省、自治区、直辖市人民政府及其负责管理地震工作的机构和国务院地震工作主管部门报告。
北京市的地震短期预报和临震预报,由国务院地震工作主管部门和北京市人民政府负责地震工作的机构,组织召开地震震情会商会,提出地震预报意见,经国务院地震工作主管部门组织评审后,报国务院批准,由北京市人民政府发布。


\section{47.什么是地震谣传?}
\label{\detokenize{index:id53}}
答:有时,会有一些关于地震的“消息”在社会上流传,它们并非是政府公开发布的地震预报意见,而是地震谣传。
强烈地震灾害造成人们对地震的恐惧,加之对地震知识和相关法规不够了解,人们便容易偏听偏信一些无根据的、所谓的“地震消息”,这是地震谣传得以存在的土壤。产生地震谣传的具体原因有:
①把一些自然现象,如由于气候返暖果树二次开花,春季大地复苏解冻而引起的翻砂、冒水等现象,误认为是地震异常。
②地震部门正常的业务活动,如野外观测、地震考察、对某种前兆异常的落实、地震会商、抗震会议、防震减灾宣传等,引起的猜疑。
③来自海外蛊惑人心的宣传,或别有用心的造谣。
④受封建迷信思想的蒙蔽而上当受骗。


\section{48.怎样识别地震谣传?}
\label{\detokenize{index:id54}}
答:以下几种情况可以判定是地震谣传:
①超过目前预报的实际水平,三要素十分“精确”的所谓地震预报意见。如传闻中地震发生的时间、地点非常具体,甚至发震时间精确到“上午”、“晚上”。
②跨国地震预报。如果传说地震是由外国人预报的,那肯定是谣传,因为这既不符合我国关于发布地震预报的规定,也不符合国际间的约定。
③对地震后果过分渲染的传言。有时,特别是强震发生后常会出现“某个地方将要下陷”“某个地方要遭水淹”等等传言,这种耸人听闻的消息也是不可信的。


\section{49.听到地震谣传怎么办?}
\label{\detokenize{index:id55}}
答:①不相信。尽管地震预测尚未过关,但是有地震部门在进行监测研究,有政府部门在组织和部署有关防震减灾工作,因此不要相信毫无科学依据的地震谣传。
②不传播。应当相信,只要政府知道破坏性地震将要发生,是绝对不会向人民群众隐瞒的。因此如果听到地震谣传,千万不要继续传播。
③及时报告。当听到地震传闻时,要及时向当地政府和地震部门反映,协助地震部门平息谣传。
④如果发现动物、植物或地下水异常时,要及时向地震部门报告,不要随意散布,地震部门会采取措施及时进行调查核实。



\renewcommand{\indexname}{索引}
\printindex
\end{document}